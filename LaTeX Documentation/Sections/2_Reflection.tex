\section{Reflection}
\label{sec:reflection}

% \textbf{In 3-5 pages, 1500-2000 words}\\

% The purpose of the reflection is to show that you can reflect and assess your own work and learning process critically.\\

% This section needs to be adjusted to align with the reflection requirements specified in the selected task.\\

% \textbf{Note:} You should address all the questions from your selected task. Please list each question and provide your answers in the following enumeration.\\

% For example:
\begin{enumerate}
    \item   What was the most interesting thing that you learnt while working on the portfolio?
            What aspects did you find interesting or surprising? 
    
    \textbf{Answer:}
    1. The most interesting thing is that I am curious about how the big model is fine\-tuned and the results of the fine-tuning.
    2. What impressed me most was that fine\-tuning can really improve the performance of large models, 
    and even a slight increase of a few hundred pieces of data can clearly and intuitively show the performance improvement from the numbers.
    
    \item   Which part of the portfolio are you (most) proud of?
            Why?
            What were the challenges you faced, and how did you overcome them? 

    \textbf{Answer:}
    I am most proud of the fact that I started from the sixth step of synthesizing the dataset and finished the twelfth step. 
    Because I had an exam this morning, and I was frustrated that my score dropped instead of rising 
    because I was stuck in the fifth step of fine-tuning a few days ago. 
    I still persisted after the exam this morning until 11:30 pm. 
    I wanted to give up countless times, and if it really didn't work, I would try again next semester, but I still persisted until now.

    \item   What adjustments to your design and implementation were necessary during the implementation phase?
            What would you change or do differently if you had to do the portfolio task a second time?
            What would be potential areas for future improvement?
    
    \textbf{Answer:}
    1. During the implementation phase, I changed Model A from a normal version to an instruction\-funed version.
    2. If I had the chance to do it again or allocate more time to the portfolio earlier, 
    I would really like to figure out what the difference is between the regular version of llama and 
    the instruction-tuned version of llama that causes the instruction-tuned version of llama 
    to support pipeline messages while the regular version of llama does not.
    And when the implementation was really fine\-tuned, I realized that the translation type is a seq2seq task, 
    and using a large seq2seq model may have better results. However, llama is of the casual\_LM type, 
    so I have to use the instruction version to convert the translation task into a reasoning task about translation.
    3. In the future, I can increase the amount of data, clean the data, and improve the quality of the data. 
    I can also choose several more metrics to judge the quality of translation in multiple dimensions. 
    BLEU only focuses on the accuracy of words and is suitable for short sentences. 
    I can also increase the length of the translated sentences, 
    because the data set is related to the European Parliament, and most of the written sentences are formal and long sentences.

    \item   Include a brief section on ethical considerations when using these models on language translation tasks.
    
    \textbf{Answer:}
    See Section~\ref{sec:ethics} on page~\pageref{sec:ethics}.

    \item   From the lecture/course including guest lectures, what topic excited you the most?
            Why?
            What would you like to learn more about and why?
    
    \textbf{Answer:}
    1. I am very interested in multi-agent in exercise.
    2. Because I find it interesting to observe the reasoning dialogues between agents, 
    and I am curious how they would debate in more difficult situations, such as the trolley problem.
    3. I also want to learn more about the natural interaction between humans and AI agents, 
    such as intent understanding and trust building, because I want to see how agents build trust with people.


    \item   How did you find working with DIFY platform during the course work?
            Would you recommend using DIFY in learning Generative AI technologies and why?
            What is the best start for learning Generative AI either by Python code or No-code platforms and why?
    
    \textbf{Answer:}
    1. I don't like DIFY very much, especially in practice, many times I make mistakes and I can't find debugging information, 
    which increases the difficulty of practice. I prefer to write directly in code. 
    But for novices, this graphical platform can increase interest and is more intuitive, which is also an advantage.
    2. As mentioned in 1.
    3. As mentioned in 1.

    \item   How did you find the assignments and exercises in the course and how they help you in portfolio exam?
    
    \textbf{Answer:}
    Unfortunately, we didn't have practice fine-tuning for Task 1 in the practice class, 
    which took me some time to learn on my own, but I was happy to learn about the huggingface platform in the practice class. 
    I hope GenAI can talk more about the cookbook in huggingface in the future.
    
\end{enumerate}
